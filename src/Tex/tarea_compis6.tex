% !TEX TS-program = pdflatex
% !TEX encoding = UTF-8 Unicode

% This is a simple template for a LaTeX document using the "article" class.
% See "book", "report", "letter" for other types of document.

\documentclass[11pt]{article} % use larger type; default would be 10pt

\usepackage[utf8]{inputenc} % set input encoding (not needed with XeLaTeX)

%%% Examples of Article customizations
% These packages are optional, depending whether you want the features they provide.
% See the LaTeX Companion or other references for full information.

%%% PAGE DIMENSIONS
\usepackage{geometry} % to change the page dimensions
\geometry{a4paper} % or letterpaper (US) or a5paper or....
% \geometry{margin=2in} % for example, change the margins to 2 inches all round
% \geometry{landscape} % set up the page for landscape
%   read geometry.pdf for detailed page layout information

\usepackage{graphicx} % support the \includegraphics command and options

% \usepackage[parfill]{parskip} % Activate to begin paragraphs with an empty line rather than an indent

%%% PACKAGES
\usepackage{booktabs} % for much better looking tables
\usepackage{array} % for better arrays (eg matrices) in maths
\usepackage{paralist} % very flexible & customisable lists (eg. enumerate/itemize, etc.)
\usepackage{verbatim} % adds environment for commenting out blocks of text & for better verbatim
\usepackage{subfig} % make it possible to include more than one captioned figure/table in a single float
\usepackage{amsmath}
% These packages are all incorporated in the memoir class to one degree or another...

%%% HEADERS & FOOTERS
\usepackage{fancyhdr} % This should be set AFTER setting up the page geometry
\pagestyle{fancy} % options: empty , plain , fancy
\renewcommand{\headrulewidth}{0pt} % customise the layout...
\lhead{}\chead{}\rhead{}
\lfoot{}\cfoot{\thepage}\rfoot{}

%%% SECTION TITLE APPEARANCE
%\usepackage{sectsty}
%\allsectionsfont{\sffamily\mdseries\upshape} % (See the fntguide.pdf for font help)
% (This matches ConTeXt defaults)

%%% ToC (table of contents) APPEARANCE
\usepackage[nottoc,notlof,notlot]{tocbibind} % Put the bibliography in the ToC
\usepackage[titles,subfigure]{tocloft} % Alter the style of the Table of Contents
\renewcommand{\cftsecfont}{\rmfamily\mdseries\upshape}
\renewcommand{\cftsecpagefont}{\rmfamily\mdseries\upshape} % No bold!

%%% END Article customizations

%%% The "real" document content comes below...

\title{Compiladores: Tarea 6}
\author{Eguiarte Morett Luis Andres}
\date{Octubre 11, 2017} % Activate to display a given date or no date (if empty),
         % otherwise the current date is printed 

\begin{document}
\maketitle

\section{Autómata LR(0) y tabla de análisis sintáctico.}

Obtener la el autómata y la tabla de análisis sintáctico Lr(0) para la siguiente gramática.

\subsection{$A\rightarrow(A)\vert a$}

Extendiendola.
\\

$A'\rightarrow A$ \\
$A\rightarrow (A)\vert a$ \\ 

El autómata queda:\\

$I_0$ =\{$A'\rightarrow \bullet A$ \\
      	\hspace{3cm}$A\rightarrow \bullet (A)$ \\
	\hspace{3cm}$A\rightarrow \bullet a$ \} \\

goto($I_0$, A) = \{ $A'\rightarrow A \bullet$ \} = $I_1$ \\

goto($I_0$, ( ) = \{ $A\rightarrow (\bullet A)$ \\
			$A \rightarrow \bullet a$ \} = $I_2$ \\

goto($I_0$, a ) = \{ $A \rightarrow a\bullet$ \} = $I_3$ \\

goto($I_2$, A) = \{ $A\rightarrow (A \bullet)$ \} =  $I_4$ \\

goto($I_2$, a) = $I_3$ \\

goto($I_4$, ) ) = \{ $A\rightarrow (A)\bullet$ \} =  $I_5$ \\
\\
Follow(A) = \{ \$ ) \}
\\
La tabla de análisis sintáctico LR(0) queda: \\
\begin{tabular}{l|*{6}{c}r}
                  & ( & ) & a & \$ & A \\
\hline
0 & d2 &  & d3 &   &  1 \\
1 &  &   &   & acepta  &  \\
2 &   &   & d3 &   & 4   \\
3  &  & r2 &  & r2 &  \\
4  &   & d5  &  &   &  \\
5  &   & r1 &  & r1 &  \\
\end{tabular}

\end{document}
