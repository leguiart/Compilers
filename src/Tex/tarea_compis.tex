% !TEX TS-program = pdflatex
% !TEX encoding = UTF-8 Unicode

% This is a simple template for a LaTeX document using the "article" class.
% See "book", "report", "letter" for other types of document.

\documentclass[11pt]{article} % use larger type; default would be 10pt

\usepackage[utf8]{inputenc} % set input encoding (not needed with XeLaTeX)

%%% Examples of Article customizations
% These packages are optional, depending whether you want the features they provide.
% See the LaTeX Companion or other references for full information.

%%% PAGE DIMENSIONS
\usepackage{geometry} % to change the page dimensions
\geometry{a4paper} % or letterpaper (US) or a5paper or....
% \geometry{margin=2in} % for example, change the margins to 2 inches all round
% \geometry{landscape} % set up the page for landscape
%   read geometry.pdf for detailed page layout information

\usepackage{graphicx} % support the \includegraphics command and options

% \usepackage[parfill]{parskip} % Activate to begin paragraphs with an empty line rather than an indent

%%% PACKAGES
\usepackage{booktabs} % for much better looking tables
\usepackage{array} % for better arrays (eg matrices) in maths
\usepackage{paralist} % very flexible & customisable lists (eg. enumerate/itemize, etc.)
\usepackage{verbatim} % adds environment for commenting out blocks of text & for better verbatim
\usepackage{subfig} % make it possible to include more than one captioned figure/table in a single float
\usepackage{amsmath}
% These packages are all incorporated in the memoir class to one degree or another...

%%% HEADERS & FOOTERS
\usepackage{fancyhdr} % This should be set AFTER setting up the page geometry
\pagestyle{fancy} % options: empty , plain , fancy
\renewcommand{\headrulewidth}{0pt} % customise the layout...
\lhead{}\chead{}\rhead{}
\lfoot{}\cfoot{\thepage}\rfoot{}

%%% SECTION TITLE APPEARANCE
%\usepackage{sectsty}
%\allsectionsfont{\sffamily\mdseries\upshape} % (See the fntguide.pdf for font help)
% (This matches ConTeXt defaults)

%%% ToC (table of contents) APPEARANCE
\usepackage[nottoc,notlof,notlot]{tocbibind} % Put the bibliography in the ToC
\usepackage[titles,subfigure]{tocloft} % Alter the style of the Table of Contents
\renewcommand{\cftsecfont}{\rmfamily\mdseries\upshape}
\renewcommand{\cftsecpagefont}{\rmfamily\mdseries\upshape} % No bold!

%%% END Article customizations

%%% The "real" document content comes below...

\title{Compiladores: Tarea 5}
\author{Eguiarte Morett Luis Andres}
\date{Octubre 5, 2017} % Activate to display a given date or no date (if empty),
         % otherwise the current date is printed 

\begin{document}
\maketitle

\section{Tabla de análisis sintáctico LL(1)}

Obtener las tablas de análisis sintáctico LL(1) para las siguientes gramáticas.

\subsection{$S\rightarrow*SS\vert+SS\vert a$}

La gramática ya es no ambigua y libre de recursividad por la izquierda, proseguimos a sacar First() y Follow().
\\

First(S) = \{* + a \} \\
Follow(S) = \{ \$ \} \\

La tabla LL(1) queda: \\

\begin{tabular}{l*{6}{c}r}
              & * & + & a & \$ \\
\hline
S & 1 & 2 & 3 &     \\
\end{tabular}

\subsection{$S\rightarrow SS*\vert SS+\vert a$}

Factorizando por la izquierda: \\
$S \rightarrow SSS' \vert a$ \\
$S' \rightarrow * \vert + $ \\
\\
Eliminando recursividad por la izquierda: \\
$S \rightarrow aS''$ \\
$S'' \rightarrow SS'S'' \vert \varepsilon$ \\
$S' \rightarrow * \vert +$
\\

First(S) = \{ a \} \\
First(S'') = \{ a $\varepsilon$ \} \\
First(S') =  \{ * + \} \\
\\
Follow(S) = \{ \$ * + \}\\
Follow(S'') = \{ \$ * + \}\\
Follow(S') = \{ a \$ * + \}\\
\\
La tabla de análisis sintáctico queda: \\
\begin{tabular}{l*{6}{c}r}
                  & a & * & + & \$ \\
\hline
S & 1 &   &   &   \\
S'' & 2 & 3 & 3 & 3 \\
S'  &   & 4 & 5 &   \\
\end{tabular}

\subsection{$S\rightarrow S+S\vert SS\vert S* \vert (S) \vert a$}
Eliminando ambiguedad de la gramática: \\
+ --- $E \rightarrow E + T \vert T$ \\
conc * --- $T \rightarrow T* \vert TT \vert F$ \\
() --- $F \rightarrow (E) \vert a$ \\
\\
Eliminando recursividad por la izquierda: \\
$E \rightarrow TE'$ \\
$E' \rightarrow +TE' \vert \varepsilon $ \\
$T \rightarrow FT'$ \\
$T' \rightarrow TT' \vert *T' \vert \varepsilon$ \\
$F \rightarrow (E) \vert a $ \\
\\

First(E) = First(T) = First(F) = \{ ( a \} \\
First(E') = \{ + $\varepsilon$ \} \\
First(T') = \{ ( a $\varepsilon$ \} \\
\\
Follow(E) = \{ \$ ) \} \\
Follow(E') = \{ \$ ) \} \\
Follow(T) = \{ + \$ ) \} \\
Follow(T') = \{ + \$ \} \\
Follow(F) = \{ ( a * + \$ ) \}\\
\\
La tabla de análisis sintáctico queda: \\
\begin{tabular}{l*{6}{c}r}
                  & + & * & ( & ) & a & \$ \\
\hline
E &   &   & 1 &   & 1 &   \\
E' & 2 &   &   & 3 &   & 3 \\
T &   &   & 4 &   & 4 &   \\
T'  & 7 & 6 & 5 & 7 & 5 & 7 \\
F  &   &   & 8 &   & 9 &   \\
\end{tabular}

\end{document}
